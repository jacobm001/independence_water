
%% bare_jrnl.tex
%% V1.4b
%% 2015/08/26
%% by Michael Shell
%% see http://www.michaelshell.org/
%% for current contact information.
%%
%% This is a skeleton file demonstrating the use of IEEEtran.cls
%% (requires IEEEtran.cls version 1.8b or later) with an IEEE
%% journal paper.
%%
%% Support sites:
%% http://www.michaelshell.org/tex/ieeetran/
%% http://www.ctan.org/pkg/ieeetran
%% and
%% http://www.ieee.org/

%%*************************************************************************
%% Legal Notice:
%% This code is offered as-is without any warranty either expressed or
%% implied; without even the implied warranty of MERCHANTABILITY or
%% FITNESS FOR A PARTICULAR PURPOSE! 
%% User assumes all risk.
%% In no event shall the IEEE or any contributor to this code be liable for
%% any damages or losses, including, but not limited to, incidental,
%% consequential, or any other damages, resulting from the use or misuse
%% of any information contained here.
%%
%% All comments are the opinions of their respective authors and are not
%% necessarily endorsed by the IEEE.
%%
%% This work is distributed under the LaTeX Project Public License (LPPL)
%% ( http://www.latex-project.org/ ) version 1.3, and may be freely used,
%% distributed and modified. A copy of the LPPL, version 1.3, is included
%% in the base LaTeX documentation of all distributions of LaTeX released
%% 2003/12/01 or later.
%% Retain all contribution notices and credits.
%% ** Modified files should be clearly indicated as such, including  **
%% ** renaming them and changing author support contact information. **
%%*************************************************************************


% *** Authors should verify (and, if needed, correct) their LaTeX system  ***
% *** with the testflow diagnostic prior to trusting their LaTeX platform ***
% *** with production work. The IEEE's font choices and paper sizes can   ***
% *** trigger bugs that do not appear when using other class files.       ***                          ***
% The testflow support page is at:
% http://www.michaelshell.org/tex/testflow/



\documentclass[journal]{IEEEtran}
%
% If IEEEtran.cls has not been installed into the LaTeX system files,
% manually specify the path to it like:
% \documentclass[journal]{../sty/IEEEtran}





% Some very useful LaTeX packages include:
% (uncomment the ones you want to load)


% *** MISC UTILITY PACKAGES ***
%
%\usepackage{ifpdf}
% Heiko Oberdiek's ifpdf.sty is very useful if you need conditional
% compilation based on whether the output is pdf or dvi.
% usage:
% \ifpdf
%   % pdf code
% \else
%   % dvi code
% \fi
% The latest version of ifpdf.sty can be obtained from:
% http://www.ctan.org/pkg/ifpdf
% Also, note that IEEEtran.cls V1.7 and later provides a builtin
% \ifCLASSINFOpdf conditional that works the same way.
% When switching from latex to pdflatex and vice-versa, the compiler may
% have to be run twice to clear warning/error messages.

\usepackage{float}




% *** CITATION PACKAGES ***
%
%\usepackage{cite}
% cite.sty was written by Donald Arseneau
% V1.6 and later of IEEEtran pre-defines the format of the cite.sty package
% \cite{} output to follow that of the IEEE. Loading the cite package will
% result in citation numbers being automatically sorted and properly
% "compressed/ranged". e.g., [1], [9], [2], [7], [5], [6] without using
% cite.sty will become [1], [2], [5]--[7], [9] using cite.sty. cite.sty's
% \cite will automatically add leading space, if needed. Use cite.sty's
% noadjust option (cite.sty V3.8 and later) if you want to turn this off
% such as if a citation ever needs to be enclosed in parenthesis.
% cite.sty is already installed on most LaTeX systems. Be sure and use
% version 5.0 (2009-03-20) and later if using hyperref.sty.
% The latest version can be obtained at:
% http://www.ctan.org/pkg/cite
% The documentation is contained in the cite.sty file itself.






% *** GRAPHICS RELATED PACKAGES ***
%
\ifCLASSINFOpdf
   \usepackage[pdftex]{graphicx}
  % declare the path(s) where your graphic files are
  % \graphicspath{{../pdf/}{../jpeg/}}
  % and their extensions so you won't have to specify these with
  % every instance of \includegraphics
  % \DeclareGraphicsExtensions{.pdf,.jpeg,.png}
\else
  % or other class option (dvipsone, dvipdf, if not using dvips). graphicx
  % will default to the driver specified in the system graphics.cfg if no
  % driver is specified.
   \usepackage[dvips]{graphicx}
  % declare the path(s) where your graphic files are
  % \graphicspath{{../eps/}}
  % and their extensions so you won't have to specify these with
  % every instance of \includegraphics
  % \DeclareGraphicsExtensions{.eps}
\fi
% graphicx was written by David Carlisle and Sebastian Rahtz. It is
% required if you want graphics, photos, etc. graphicx.sty is already
% installed on most LaTeX systems. The latest version and documentation
% can be obtained at: 
% http://www.ctan.org/pkg/graphicx
% Another good source of documentation is "Using Imported Graphics in
% LaTeX2e" by Keith Reckdahl which can be found at:
% http://www.ctan.org/pkg/epslatex
%
% latex, and pdflatex in dvi mode, support graphics in encapsulated
% postscript (.eps) format. pdflatex in pdf mode supports graphics
% in .pdf, .jpeg, .png and .mps (metapost) formats. Users should ensure
% that all non-photo figures use a vector format (.eps, .pdf, .mps) and
% not a bitmapped formats (.jpeg, .png). The IEEE frowns on bitmapped formats
% which can result in "jaggedy"/blurry rendering of lines and letters as
% well as large increases in file sizes.
%
% You can find documentation about the pdfTeX application at:
% http://www.tug.org/applications/pdftex





% *** MATH PACKAGES ***
%
%\usepackage{amsmath}
% A popular package from the American Mathematical Society that provides
% many useful and powerful commands for dealing with mathematics.
%
% Note that the amsmath package sets \interdisplaylinepenalty to 10000
% thus preventing page breaks from occurring within multiline equations. Use:
%\interdisplaylinepenalty=2500
% after loading amsmath to restore such page breaks as IEEEtran.cls normally
% does. amsmath.sty is already installed on most LaTeX systems. The latest
% version and documentation can be obtained at:
% http://www.ctan.org/pkg/amsmath





% *** SPECIALIZED LIST PACKAGES ***
%
%\usepackage{algorithmic}
% algorithmic.sty was written by Peter Williams and Rogerio Brito.
% This package provides an algorithmic environment fo describing algorithms.
% You can use the algorithmic environment in-text or within a figure
% environment to provide for a floating algorithm. Do NOT use the algorithm
% floating environment provided by algorithm.sty (by the same authors) or
% algorithm2e.sty (by Christophe Fiorio) as the IEEE does not use dedicated
% algorithm float types and packages that provide these will not provide
% correct IEEE style captions. The latest version and documentation of
% algorithmic.sty can be obtained at:
% http://www.ctan.org/pkg/algorithms
% Also of interest may be the (relatively newer and more customizable)
% algorithmicx.sty package by Szasz Janos:
% http://www.ctan.org/pkg/algorithmicx




% *** ALIGNMENT PACKAGES ***
%
%\usepackage{array}
% Frank Mittelbach's and David Carlisle's array.sty patches and improves
% the standard LaTeX2e array and tabular environments to provide better
% appearance and additional user controls. As the default LaTeX2e table
% generation code is lacking to the point of almost being broken with
% respect to the quality of the end results, all users are strongly
% advised to use an enhanced (at the very least that provided by array.sty)
% set of table tools. array.sty is already installed on most systems. The
% latest version and documentation can be obtained at:
% http://www.ctan.org/pkg/array


% IEEEtran contains the IEEEeqnarray family of commands that can be used to
% generate multiline equations as well as matrices, tables, etc., of high
% quality.




% *** SUBFIGURE PACKAGES ***
%\ifCLASSOPTIONcompsoc
%  \usepackage[caption=false,font=normalsize,labelfont=sf,textfont=sf]{subfig}
%\else
%  \usepackage[caption=false,font=footnotesize]{subfig}
%\fi
% subfig.sty, written by Steven Douglas Cochran, is the modern replacement
% for subfigure.sty, the latter of which is no longer maintained and is
% incompatible with some LaTeX packages including fixltx2e. However,
% subfig.sty requires and automatically loads Axel Sommerfeldt's caption.sty
% which will override IEEEtran.cls' handling of captions and this will result
% in non-IEEE style figure/table captions. To prevent this problem, be sure
% and invoke subfig.sty's "caption=false" package option (available since
% subfig.sty version 1.3, 2005/06/28) as this is will preserve IEEEtran.cls
% handling of captions.
% Note that the Computer Society format requires a larger sans serif font
% than the serif footnote size font used in traditional IEEE formatting
% and thus the need to invoke different subfig.sty package options depending
% on whether compsoc mode has been enabled.
%
% The latest version and documentation of subfig.sty can be obtained at:
% http://www.ctan.org/pkg/subfig




% *** FLOAT PACKAGES ***
%
%\usepackage{fixltx2e}
% fixltx2e, the successor to the earlier fix2col.sty, was written by
% Frank Mittelbach and David Carlisle. This package corrects a few problems
% in the LaTeX2e kernel, the most notable of which is that in current
% LaTeX2e releases, the ordering of single and double column floats is not
% guaranteed to be preserved. Thus, an unpatched LaTeX2e can allow a
% single column figure to be placed prior to an earlier double column
% figure.
% Be aware that LaTeX2e kernels dated 2015 and later have fixltx2e.sty's
% corrections already built into the system in which case a warning will
% be issued if an attempt is made to load fixltx2e.sty as it is no longer
% needed.
% The latest version and documentation can be found at:
% http://www.ctan.org/pkg/fixltx2e


%\usepackage{stfloats}
% stfloats.sty was written by Sigitas Tolusis. This package gives LaTeX2e
% the ability to do double column floats at the bottom of the page as well
% as the top. (e.g., "\begin{figure*}[!b]" is not normally possible in
% LaTeX2e). It also provides a command:
%\fnbelowfloat
% to enable the placement of footnotes below bottom floats (the standard
% LaTeX2e kernel puts them above bottom floats). This is an invasive package
% which rewrites many portions of the LaTeX2e float routines. It may not work
% with other packages that modify the LaTeX2e float routines. The latest
% version and documentation can be obtained at:
% http://www.ctan.org/pkg/stfloats
% Do not use the stfloats baselinefloat ability as the IEEE does not allow
% \baselineskip to stretch. Authors submitting work to the IEEE should note
% that the IEEE rarely uses double column equations and that authors should try
% to avoid such use. Do not be tempted to use the cuted.sty or midfloat.sty
% packages (also by Sigitas Tolusis) as the IEEE does not format its papers in
% such ways.
% Do not attempt to use stfloats with fixltx2e as they are incompatible.
% Instead, use Morten Hogholm'a dblfloatfix which combines the features
% of both fixltx2e and stfloats:
%
% \usepackage{dblfloatfix}
% The latest version can be found at:
% http://www.ctan.org/pkg/dblfloatfix




%\ifCLASSOPTIONcaptionsoff
%  \usepackage[nomarkers]{endfloat}
% \let\MYoriglatexcaption\caption
% \renewcommand{\caption}[2][\relax]{\MYoriglatexcaption[#2]{#2}}
%\fi
% endfloat.sty was written by James Darrell McCauley, Jeff Goldberg and 
% Axel Sommerfeldt. This package may be useful when used in conjunction with 
% IEEEtran.cls'  captionsoff option. Some IEEE journals/societies require that
% submissions have lists of figures/tables at the end of the paper and that
% figures/tables without any captions are placed on a page by themselves at
% the end of the document. If needed, the draftcls IEEEtran class option or
% \CLASSINPUTbaselinestretch interface can be used to increase the line
% spacing as well. Be sure and use the nomarkers option of endfloat to
% prevent endfloat from "marking" where the figures would have been placed
% in the text. The two hack lines of code above are a slight modification of
% that suggested by in the endfloat docs (section 8.4.1) to ensure that
% the full captions always appear in the list of figures/tables - even if
% the user used the short optional argument of \caption[]{}.
% IEEE papers do not typically make use of \caption[]'s optional argument,
% so this should not be an issue. A similar trick can be used to disable
% captions of packages such as subfig.sty that lack options to turn off
% the subcaptions:
% For subfig.sty:
% \let\MYorigsubfloat\subfloat
% \renewcommand{\subfloat}[2][\relax]{\MYorigsubfloat[]{#2}}
% However, the above trick will not work if both optional arguments of
% the \subfloat command are used. Furthermore, there needs to be a
% description of each subfigure *somewhere* and endfloat does not add
% subfigure captions to its list of figures. Thus, the best approach is to
% avoid the use of subfigure captions (many IEEE journals avoid them anyway)
% and instead reference/explain all the subfigures within the main caption.
% The latest version of endfloat.sty and its documentation can obtained at:
% http://www.ctan.org/pkg/endfloat
%
% The IEEEtran \ifCLASSOPTIONcaptionsoff conditional can also be used
% later in the document, say, to conditionally put the References on a 
% page by themselves.




% *** PDF, URL AND HYPERLINK PACKAGES ***
%
%\usepackage{url}
% url.sty was written by Donald Arseneau. It provides better support for
% handling and breaking URLs. url.sty is already installed on most LaTeX
% systems. The latest version and documentation can be obtained at:
% http://www.ctan.org/pkg/url
% Basically, \url{my_url_here}.




% *** Do not adjust lengths that control margins, column widths, etc. ***
% *** Do not use packages that alter fonts (such as pslatex).         ***
% There should be no need to do such things with IEEEtran.cls V1.6 and later.
% (Unless specifically asked to do so by the journal or conference you plan
% to submit to, of course. )


% correct bad hyphenation here
\hyphenation{op-tical net-works semi-conduc-tor}


\begin{document}
\onecolumn
%
% paper title
% Titles are generally capitalized except for words such as a, an, and, as,
% at, but, by, for, in, nor, of, on, or, the, to and up, which are usually
% not capitalized unless they are the first or last word of the title.
% Linebreaks \\ can be used within to get better formatting as desired.
% Do not put math or special symbols in the title.
\title{Project Abzu: Water Usage Visualization \\
	\LARGE Midterm Report}
%
%
% author names and IEEE memberships
% note positions of commas and nonbreaking spaces ( ~ ) LaTeX will not break
% a structure at a ~ so this keeps an author's name from being broken across
% two lines.
% use \thanks{} to gain access to the first footnote area
% a separate \thanks must be used for each paragraph as LaTeX2e's \thanks
% was not built to handle multiple paragraphs
%

\author{Tyler~Fosback,
		Jacob~Mastel,
        and~Alexander~Merrill
        }% <-this % stops a space
%\thanks{M. Shell was with the Department
%of Electrical and Computer Engineering, Georgia Institute of Technology, Atlanta,
%GA, 30332 USA e-mail: (see http://www.michaelshell.org/contact.html).}% <-this % stops a space
%\thanks{J. Doe and J. Doe are with Anonymous University.}% <-this % stops a space
%\thanks{Manuscript received April 19, 2005; revised August 26, 2015.}

% note the % following the last \IEEEmembership and also \thanks - 
% these prevent an unwanted space from occurring between the last author name
% and the end of the author line. i.e., if you had this:
% 
% \author{....lastname \thanks{...} \thanks{...} }
%                     ^------------^------------^----Do not want these spaces!
%
% a space would be appended to the last name and could cause every name on that
% line to be shifted left slightly. This is one of those "LaTeX things". For
% instance, "\textbf{A} \textbf{B}" will typeset as "A B" not "AB". To get
% "AB" then you have to do: "\textbf{A}\textbf{B}"
% \thanks is no different in this regard, so shield the last } of each \thanks
% that ends a line with a % and do not let a space in before the next \thanks.
% Spaces after \IEEEmembership other than the last one are OK (and needed) as
% you are supposed to have spaces between the names. For what it is worth,
% this is a minor point as most people would not even notice if the said evil
% space somehow managed to creep in.
\begin{titlepage}

\maketitle

\begin{abstract}

The City of Independence wants a mobile web application enabling city residents to view water usage. Users should be able to see usage on a city scale and on their home and business meters. The application should be available on phones, tablets, personal computers, and a kiosk at city hall. 
\end{abstract}

\end{titlepage}


% The paper headers
\markboth{Oregon State Univeristy,~CS~462, Group~31, Abzu, March~2016}%
{Shell \MakeLowercase{\textit{et al.}}: Bare Demo of IEEEtran.cls for IEEE Journals}
% The only time the second header will appear is for the odd numbered pages
% after the title page when using the twoside option.
% 
% *** Note that you probably will NOT want to include the author's ***
% *** name in the headers of peer review papers.                   ***
% You can use \ifCLASSOPTIONpeerreview for conditional compilation here if
% you desire.




% If you want to put a publisher's ID mark on the page you can do it like
% this:
%\IEEEpubid{0000--0000/00\$00.00~\copyright~2015 IEEE}
% Remember, if you use this you must call \IEEEpubidadjcol in the second
% column for its text to clear the IEEEpubid mark.



% use for special paper notices
%\IEEEspecialpapernotice{(Invited Paper)}




% make the title area
%\maketitle
\tableofcontents
\newpage

% As a general rule, do not put math, special symbols or citations
% in the abstract or keywords.
%\begin{abstract}
%The abstract goes here.
%\end{abstract}

% Note that keywords are not normally used for peerreview papers.
%\begin{IEEEkeywords}
%IEEE, IEEEtran, journal, \LaTeX, paper, template.
%\end{IEEEkeywords}






% For peer review papers, you can put extra information on the cover
% page as needed:
% \ifCLASSOPTIONpeerreview
% \begin{center} \bfseries EDICS Category: 3-BBND \end{center}
% \fi
%
% For peerreview papers, this IEEEtran command inserts a page break and
% creates the second title. It will be ignored for other modes.
\IEEEpeerreviewmaketitle



\section{Overview}
%Briefly recaps the project purposes and goals

% The very first letter is a 2 line initial drop letter followed
% by the rest of the first word in caps.
% 
% form to use if the first word consists of a single letter:
% \IEEEPARstart{A}{demo} file is ....
% 
% form to use if you need the single drop letter followed by
% normal text (unknown if ever used by the IEEE):
% \IEEEPARstart{A}{}demo file is ....
% 
% Some journals put the first two words in caps:
% \IEEEPARstart{T}{his demo} file is ....
% 
% Here we have the typical use of a "T" for an initial drop letter
% and "HIS" in caps to complete the first word.
%\IEEEPARstart{T}{his} demo file is intended to serve as a ``starter file'' for IEEE journal papers produced under \LaTeX\ using IEEEtran.cls version 1.8b and later.
% You must have at least 2 lines in the paragraph with the drop letter
% (should never be an issue)
%I wish you the best of success.

%\hfill mds
 

\subsection{Scope}
\IEEEPARstart{T}{his} software system will be an online database and web application for the City of Independence and its citizens. The system should be able to display water usage on an individual and aggregated basis, compare individual usage to average usage, and raise warnings about possible leaks. Usage will be displayed as numbers and as a graphical representation of the data. 

\subsection{Purpose}
\IEEEPARstart{T}{he} purpose of this document is to present a detailed description of progress on Abzu. This document will present an overview of Abzu, explain what progress has been made on Abzu, explain what remains to be implemented for Abzu, and present problems and challenges encountered while implementing Abzu.

\subsection{Intended Audience}

\IEEEPARstart{T}{his} document is written for the developers, D. Kevin McGrath, and Xinze Guan. 




% needed in second column of first page if using \IEEEpubid
%\IEEEpubidadjcol



% An example of a floating figure using the graphicx package.
% Note that \label must occur AFTER (or within) \caption.
% For figures, \caption should occur after the \includegraphics.
% Note that IEEEtran v1.7 and later has special internal code that
% is designed to preserve the operation of \label within \caption
% even when the captionsoff option is in effect. However, because
% of issues like this, it may be the safest practice to put all your
% \label just after \caption rather than within \caption{}.
%
% Reminder: the "draftcls" or "draftclsnofoot", not "draft", class
% option should be used if it is desired that the figures are to be
% displayed while in draft mode.
%
%\begin{figure}[!t]
%\centering
%\includegraphics[width=2.5in]{myfigure}
% where an .eps filename suffix will be assumed under latex, 
% and a .pdf suffix will be assumed for pdflatex; or what has been declared
% via \DeclareGraphicsExtensions.
%\caption{Simulation results for the network.}
%\label{fig_sim}
%\end{figure}

% Note that the IEEE typically puts floats only at the top, even when this
% results in a large percentage of a column being occupied by floats.


% An example of a double column floating figure using two subfigures.
% (The subfig.sty package must be loaded for this to work.)
% The subfigure \label commands are set within each subfloat command,
% and the \label for the overall figure must come after \caption.
% \hfil is used as a separator to get equal spacing.
% Watch out that the combined width of all the subfigures on a 
% line do not exceed the text width or a line break will occur.
%
%\begin{figure*}[!t]
%\centering
%\subfloat[Case I]{\includegraphics[width=2.5in]{box}%
%\label{fig_first_case}}
%\hfil
%\subfloat[Case II]{\includegraphics[width=2.5in]{box}%
%\label{fig_second_case}}
%\caption{Simulation results for the network.}
%\label{fig_sim}
%\end{figure*}
%
% Note that often IEEE papers with subfigures do not employ subfigure
% captions (using the optional argument to \subfloat[]), but instead will
% reference/describe all of them (a), (b), etc., within the main caption.
% Be aware that for subfig.sty to generate the (a), (b), etc., subfigure
% labels, the optional argument to \subfloat must be present. If a
% subcaption is not desired, just leave its contents blank,
% e.g., \subfloat[].


% An example of a floating table. Note that, for IEEE style tables, the
% \caption command should come BEFORE the table and, given that table
% captions serve much like titles, are usually capitalized except for words
% such as a, an, and, as, at, but, by, for, in, nor, of, on, or, the, to
% and up, which are usually not capitalized unless they are the first or
% last word of the caption. Table text will default to \footnotesize as
% the IEEE normally uses this smaller font for tables.
% The \label must come after \caption as always.
%
%\begin{table}[!t]
%% increase table row spacing, adjust to taste
%\renewcommand{\arraystretch}{1.3}
% if using array.sty, it might be a good idea to tweak the value of
% \extrarowheight as needed to properly center the text within the cells
%\caption{An Example of a Table}
%\label{table_example}
%\centering
%% Some packages, such as MDW tools, offer better commands for making tables
%% than the plain LaTeX2e tabular which is used here.
%\begin{tabular}{|c||c|}
%\hline
%One & Two\\
%\hline
%Three & Four\\
%\hline
%\end{tabular}
%\end{table}


% Note that the IEEE does not put floats in the very first column
% - or typically anywhere on the first page for that matter. Also,
% in-text middle ("here") positioning is typically not used, but it
% is allowed and encouraged for Computer Society conferences (but
% not Computer Society journals). Most IEEE journals/conferences use
% top floats exclusively. 
% Note that, LaTeX2e, unlike IEEE journals/conferences, places
% footnotes above bottom floats. This can be corrected via the
% \fnbelowfloat command of the stfloats package.


\section{Definitions, Acronyms, and Abbreviations}

\begin{enumerate}

\item \textbf{Developers}: Tyler Fosback, Jacob Mastel, and Alexander Merrill 
\item \textbf{Client:} The City of Independence, Oregon 
\item \textbf{Stakeholders:} The City of Independence, Oregon
\item \textbf{Abzu:} Name for the Water Visualization project for the City of Independence, Oregon 
\item \textbf{Meter:} Existing devices connected to each customer’s water source.  Gives a readout of water usage in gallons. 
\item \textbf{Gateway:} An embedded Linux machine with a variety of radios. 
\item \textbf{Application Server:} The machine or system hosting the User Interface. The Application Server also requests and processes data from the Database when it receives requests from the User Interface. 
\item \textbf{User Interface:} The public facing application users interact with. 
\item \textbf{Database:} A program meant to store and retrieve relational data. 
\item \textbf{Customer:} Property stewards within the City of Independence, who have a connection to the City's water grid. 
\item \textbf{Notification:} A communication to one or more users generated by the application server. 
\item \textbf{Message:} A communication to one or more users generated by an administrator. 
\item \textbf{User:} A person who interacts with the application. 
\item \textbf{Administrator:} A user appointed by the client that receives elevated privileges. This includes all functionality defined in User Class 2 - The City. 
\item \textbf{Layer:} A high level subsystem  

\end{enumerate}


\section{Project Purposes and Goals}
\IEEEPARstart{A}{bzu} allows users to monitor water usage in tabular and graphical form. Usage data is processed to find average water usage for individual meters, for neighborhoods, and for the entire city. Abzu also detects water leaks and calculates peak usage. 

Abzu is composed of a number of discrete systems. The client has Master Meter wireless water meters currently installed to collect all water usage data for the city. The client gathers this data on a monthly basis for billing. The raw usage data is uploaded by the client into Abzu's database, hosted on DigitalOcean.

The user inputs requests into the application front-end which signals the application back-end to query the database server. The application back-end then processes the data as needed and returns the processed data to the front end. The front end uses the processed data to generate the display for the user. 



\section{Current Progress}
\subsection{Front End}
\IEEEPARstart{T}{he} User Interface is the focus of the front end so the first thing that had to be done was to design the layout. Visio was initially used to create non-functional prototypes of Abzu's UI. Visio allowed the developers to experiment and quickly look at different layout options so the best could be chosen. Information that most users would be interested in was to be displayed on every page. This includes current monthly water use, yearly water use to date, and average daily water use.

The idea of displaying this information as plain text was quickly dismissed as part of the purpose of Abzu is to be visually interesting. Options for how to visually represent the information in a clear way were then explored and D3 had an example visualization that put numbers in "liquid gauges". A water fill graphic with clear numbers is a perfect fit for Abzu so it was put into the prototype.

Next was presenting the charts and tables for the user. The first problem to be solved was navigating between data. A goal was to have Abzu load completely in one page so that users didn't have to load and reload pages every time they wanted to see different data. A panel is a familiar place to put the data and allow users to switch what is displayed there. Now navigating between data within that panel had to be sorted out.

Plain text links were immediately dismissed for the lack of visual queues. Dropdown menus were toyed with but it traded extra clicks for saving a small amount of screen space. The developers felt that screen space was not at enough of a premium to make this trade and were concerned that it didn't show the user all of their options at first glance.

With dropdown menus dismissed, buttons and tabs rose to the forefront. Both are commonly used to navigate webpages, so they would be familiar to most users. Tabs were chosen since they are able to visually connect the data of a specific panel to the navigation option. Figure \ref{fig:daygraph} shows the layout described.

\begin{figure}[H]
  \includegraphics[width=0.5\linewidth]{Per_Day_Graph.png}
  \caption{Prototype of UI demonstrating the page layout. The left column is displayed on all pages. The panel on the right contains daily water use.}
  \label{fig:daygraph}
\end{figure}

After determining the layout, how the data was to be presented had to be decided. In looking through various graphical representations, the developers determined that a bar chart and line chart best visualized the data. Bar charts are used to display non-cumulative data, such as water use for a day, while line charts are used to display cumulative data, such as the running total of daily water use in a month. Figure \ref{fig:daygraph} is an example of a bar chart for water use per day while Figure \ref{fig:daytotal} is an example of a line chart for running water total.

\begin{figure}[H]
  \includegraphics[width=0.5\linewidth]{Daily_Total_Graph.png}
  \caption{Prototype of UI demonstrating the page layout. The left column is displayed on all pages. The panel on the right contains the daily running total of water use. Note the that the running total is reset for the beginning of each month, hence the drop off.}
  \label{fig:daytotal}
\end{figure}

Now an authentication screen needed to be prototyped. Since authentication screens tend to follow a similar basic style, the developers saw no reason to reinvent the wheel and went with a standard layout. Figure \ref{fig:logon} is the prototype for the authentication screen.

\begin{figure}[H]
  \includegraphics[width=0.33\linewidth]{Log_In_Screen.png}
  \caption{Prototype of UI demonstrating the authentication screen.}
  \label{fig:logon}
\end{figure}

Last was to prototype the stretch goal of a heat map of the Independence, Oregon. As this was a stretch goal, a placeholder image was found and used. Figure \ref{fig:heatmap} is the heat map prototype.

\begin{figure}[H]
  \includegraphics[width=0.5\linewidth]{Daily_Map.png}
  \caption{Prototype of UI demonstrating the page layout. The left column is displayed on all pages. The panel on the right contains a heat map of water use in the city.}
  \label{fig:heatmap}
\end{figure}

With the prototyping done and a design chosen, development began on writing the front end. Bootstrap was chosen for the front end framework and D3 initially as the framework for visualizations. The application was built according to the prototype with an addition of a home tab which may contain the stretch goal of posted announcements. The bar charts have been changed to combo charts with the line being the average. Figure \ref{fig:combo} is a combo chart without real data in use.

\begin{figure}[H]
  \includegraphics[width=0.5\linewidth]{Combo.png}
  \caption{Alpha version of the front end demonstrating a combo chart for daily water use and average daily water use. Note that it is placeholder data as the database and back end are not in place to use real data.}
  \label{fig:combo}
\end{figure}

\subsection{Back End and API}
A URL routing system has been prototyped. It is currently only capable of routing between several test pages on a local machine.
 	
\subsection{Database}
\subsubsection{Design}
The database has been designed and implemented in SQLite. In its current design, it stores user information, meter information, the relationship between user accounts and meters, and a periodic snapshot of meter values. The database in production currently has a historical dataset of a 6 week period for 14 meters. Each entry in the historical dataset is a daily snapshot.

While the database currently has the capacity to store basic user account information and what meters they own, none of that information is currently in the database. The City of Independence has not yet determined a method in which to populate the tables, so that will need to be addressed as future development.

\subsubsection{Data Integrity}
To ensure data integrity, Abzu tries to take several steps to prevent the system from loading garbage data. At the data interface layer, the database object checks that all fields have a valid value, and in the case of the timestamp field, that it is truly a recognized timestamp. If the data is acceptable, the timestamp is converted to SQLite's supported format, then all parameters are sanitized, and then finally submitted to the database.

The secondary data integrity check is within the database schema itself. All necessary fields have a \textit{not null} attribute, and the relationships are enforced with foreign keys. Any failure of this criteria is configured to result in a failed call that can then be addressed by the data interface layer.

\subsubsection{Preventing Data Duplication}
Abzu was designed with the idea that duplicated data may be submitted to the snapshot table, \textit{meter\_read}. This easily happen if two meter readers overlap, if historical data is loaded twice, or if a meter read and historical upload are used together. To address this problem, a constraint has been put into the database that will silently fail for the problematic row.

% if have a single appendix:
%\appendix[Proof of the Zonklar Equations]
% or
%\appendix  % for no appendix heading
% do not use \section anymore after \appendix, only \section*
% is possibly needed

% use appendices with more than one appendix
% then use \section to start each appendix
% you must declare a \section before using any
% \subsection or using \label (\appendices by itself
% starts a section numbered zero.)
%


\section{Future Development}
\subsection{Front End}
\IEEEPARstart{T}{he} front end needs the back end and the database to be complete for a number of tasks. The first is to allow the front end to to query data from the database for use in the visualizations and tables. These tables need to be implemented and the graphs and charts need to resize properly to fit the screen.

A user authentication page which queries the database for credentials also needs to be made. JavaScript prototypes of the front end have been made and it should be quick to implement once the back end is in place.

Leak detection notifications needs to be added to the home tab and use the back end to query the database and calculate if a leak is detected. These notifications will be attention grabbing.

An administrator panel also needs to be created. This panel will allow the client to upload data to the database. The client has presented two data formats, one directly from the meter the other from their billing system, which will need to be acceptable. As such, it needs to be indicated what formats are allowed in the upload portion of the administrator panel.

Completion of these tasks will result in a feature complete front end.

Stretch goals include implementing a heat map and an announcements panel. The announcements panel will be managed within the administrator panel, for use by the client. Further stretch goals include usability testing through user studies.


\subsection{Back End and API} \label{future:back}
For the backend we need to design a cron job to handle leak detection. This will periodically check the database for meters with abnormally high water usage. If high usage is detected a flag will be set in the database for display on the front end.

A URL routing system will be implemented to connect the requests from the front end to the database. This system will also be used for user and session authentication.

\subsection{Database}
%Our first priority will be to create a working interface with the data. The library PDO will be able to handle the majority of the database calling, and we'll use that to make the project transition between different database types.

%We will be looking into getting a larger sample set of data from the City of Independence. We currently have the billing export for a single month. This gives us a "gallons used" value for the month, but no further detail. We also have a daily read of one meter for a couple weeks. To make our display viable, we're trying to get at a few months worth of data for all meters in which it's available.

%The database interface could use several improvements in the future.

\section{Challenges and Solutions}
\subsection{Wireless Meters}
\IEEEPARstart{A}{bzu} originally planned to make use of Mentor Graphics' internet of things gateway to read from the Master Meter wireless water meters in use by the client. Research into the meters revealed that they operate on the 33-centimeter band (902MHz - 928MHz) which is not a band which either model of Mentor Graphics' gateways operate on. Further research showed that the meters utilize AES-256 encryption and Direct Sequence Spread Spectrum (DSSS). These two features meant that a Master Meter radio is required to get readings from the wireless meters.

Research into the radio revealed that Mentor Graphics software is required to download the meter data from the radio and that software only supported Windows x86 systems. This means that any gateway connected to a meter for the purpose of real time data collection would have to be running Windows on an x86 machine. Since the Master Meter gateways are not x86 systems and run Linux, they are incompatible with the radios.

While the developers researched the meters and radios, the client explored purchasing another Master Meter radio for Abzu. The client concluded that it was cost prohibitive to purchase a radio specifically for Abzu.

In reaching out to Master Meter, the developers and the client were given verbal cease and desists in trying to access the meters. Master Meter also informed the client that they offer a system for real time data collection which offers similar services to Abzu. For these reasons, real time data collection was removed from Abzu.

\subsection{Legal}
\IEEEPARstart{A}{bzu} was stalled for several months while Oregon State University attempted to come to a legal agreement with Mentor Graphics. During fall term, the developers struggled to get the technical information from Mentor Graphics needed to design Abzu. These communication issues were due to a combination of our corporate partners' busy schedules, international locations, and the unresolved legal agreement previously mentioned.

The impact of the legal agreement was fully understood at the start of winter term. Mentor Graphics informed the developers that technical information about the Mentor Graphics cloud infrastructure and API could not be shared until Oregon State University and Mentor Graphics reached a legal agreement. This revelation clarified the struggle to get technical details during fall term and stalled development of Abzu.

The deadline for when Abzu could move to an alternate design which did not utilize Mentor Graphic's infrastructure was in flux but eventually solidified to February 22, 2016. The developers did not receive a nondisclosure agreement by this date so Abzu morphed into its current, re-imagined form.

\subsection{Front End}
\IEEEPARstart{W}{eb} development was new to our team so the first challenge was going through a crash course in Bootstrap, JavaScript, and JQuery. Fortunately Bootstrap is easy to use so getting the shell of the page together was quick. The real difficulty was in learning how to use D3.

Implementing the visualization libraries posed some interesting challenges. D3 has a lot of visualization examples that can be used but poses some interesting problems. Getting the line and bar charts to display correctly was a challenge as they had a tendency to overwrite the data of another chart. The bigger problem was that D3 builds the charts as SVGs. Normally this would be great but the problem is that the size of the SVG is determined at the time the website is loaded and could not easily be dynamically re-sized. Due to the compressed time line for the Alpha, the decision was made to switch over to Google Charts for the graphs.

Google Charts is very easy to use and producing proper charts was quickly implemented. Currently we are debugging the charts to make sure that they dynamically re-size to fit the screen, maximizing compatibility for multiple devices.


\subsection{Database}
\IEEEPARstart{I}{n} our original plan with Mentor Graphics, we had planned on using Cassandra to store information, but we changed our plans when we lost the ability to use Mentor Graphic's infrastructure. For simplicity, we decided to replace Cassandra with SQLite3 for our database. As both are relational databases, the effort to switch from one to the other is minimal at this stage in our development. 

Admittedly, PostgreSQL or even MySQL would probably work better in a production environment. However, we decided that at this point, SQLite made more sense for our development.  

As a single file, this choice allows our team to more easily synchronize data to various development machines. Additionally, each member doesn't need to spend time learning how to setup, configure, and harden a more complicated installation. Since we're on a time crunch, we decided this was the most beneficial solution. 

To allow for future extensibility, we're using PHP's built in library, PDO. Theoretically, this should allow us to write the database portion of the backend in SQLite, and then change a couple lines of code to adapt it to PostgreSQL, MySQL, MariaDB, or any other relational database supported by PDO. 

% \appendices
% \section{Proof of the First Zonklar Equation}
% Appendix one text goes here.

% you can choose not to have a title for an appendix
% if you want by leaving the argument blank
% \section{}
% Appendix two text goes here.


% use section* for acknowledgment
% \section*{Acknowledgment}


% The authors would like to thank...


% Can use something like this to put references on a page
% by themselves when using endfloat and the captionsoff option.
\ifCLASSOPTIONcaptionsoff
  \newpage
\fi



% trigger a \newpage just before the given reference
% number - used to balance the columns on the last page
% adjust value as needed - may need to be readjusted if
% the document is modified later
%\IEEEtriggeratref{8}
% The "triggered" command can be changed if desired:
%\IEEEtriggercmd{\enlargethispage{-5in}}

% references section

% can use a bibliography generated by BibTeX as a .bbl file
% BibTeX documentation can be easily obtained at:
% http://mirror.ctan.org/biblio/bibtex/contrib/doc/
% The IEEEtran BibTeX style support page is at:
% http://www.michaelshell.org/tex/ieeetran/bibtex/
%\bibliographystyle{IEEEtran}
% argument is your BibTeX string definitions and bibliography database(s)
%\bibliography{IEEEabrv,../bib/paper}
%
% <OR> manually copy in the resultant .bbl file
% set second argument of \begin to the number of references
% (used to reserve space for the reference number labels box)
% \begin{thebibliography}{1}

% \bibitem{IEEEhowto:kopka}
% H.~Kopka and P.~W. Daly, \emph{A Guide to \LaTeX}, 3rd~ed.\hskip 1em plus
%   0.5em minus 0.4em\relax Harlow, England: Addison-Wesley, 1999.

% \end{thebibliography}

% biography section
% 
% If you have an EPS/PDF photo (graphicx package needed) extra braces are
% needed around the contents of the optional argument to biography to prevent
% the LaTeX parser from getting confused when it sees the complicated
% \includegraphics command within an optional argument. (You could create
% your own custom macro containing the \includegraphics command to make things
% simpler here.)
%\begin{IEEEbiography}[{\includegraphics[width=1in,height=1.25in,clip,keepaspectratio]{mshell}}]{Michael Shell}
% or if you just want to reserve a space for a photo:



% if you will not have a photo at all:
%\begin{IEEEbiographynophoto}{Tyler Fosback}
%Biography text here.
%\end{IEEEbiographynophoto}

%\begin{IEEEbiography}{Jacob Mastel}
%Biography text here.
%\end{IEEEbiography}

% insert where needed to balance the two columns on the last page with
% biographies
%\newpage

%\begin{IEEEbiographynophoto}{Alexander Merrill}
%Biography text here.
%\end{IEEEbiographynophoto}

% You can push biographies down or up by placing
% a \vfill before or after them. The appropriate
% use of \vfill depends on what kind of text is
% on the last page and whether or not the columns
% are being equalized.

%\vfill

% Can be used to pull up biographies so that the bottom of the last one
% is flush with the other column.
%\enlargethispage{-5in}



% that's all folks
\end{document}


